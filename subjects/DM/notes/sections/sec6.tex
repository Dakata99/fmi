\section{Тема 6}

\subsection{Функции – частични и тотални}
Нека \(X\) и \(Y\) са множества.

\begin{definition}
    \bu{Частична функция} с домейн \(X\) и кодомейн \(Y\) е всяка 
    релация \mexpr{f \subseteq X \times Y}, такава че за всяко \(x \in X\) съществува не повече от едно 
    \(y \in Y\), такова че \mexpr{(x, y) \in f}, т.е.

    \mexpr{\forall x \in X : ((\lnot \exists y \in Y : (x, y) \in f) \vee ((\exists y \in Y : (x, y) \in f) \land 
    (\forall w, z \in Y : (x, w) \in f \land (x, z) \in f \to w = z)))}
\end{definition}

\begin{definition}
    \bu{Тотална функция} с домейн \(X\) и кодомейн \(Y\) е всяка 
    релация \mexpr{f \subseteq X \times Y}, такава че за всяко \(x \in X\) съществува точно едно 
    \(y \in Y\), такова че \mexpr{(x, y) \in f}, т.е.

    \mexpr{\forall x \in X : ((\exists y \in Y : (x, y) \in f) \land 
    (\forall w, z \in Y : (x, w) \in f \land (x, z) \in f \to w = z))}
\end{definition}

При дадени \(X\) и \(Y\), тоталните функции са строго подмножество на частичните.
Всяка функция (тотална) е частична функция, но обратното не е вярно.

Формално, функция е вид релация.

\subsection{Еднозначна функция, сюрекция, биекция, обратна функция}
Нека \mexpr{f: X \to Y}. Тогава: % TODO: maybe add some diagrams
\begin{itemize}
    \item \(f\) е инекция, ако \mexpr{\forall x_1, x_2 \in X : f(x_1) = f(x_2) \to x_1 = x_2}
    \item \(f\) е сюрекция, ако \mexpr{\forall y \in Y \exists x \in X : f(x) = y}. Неформално - кодомейнът да 
    бъде "покрит" от изображинието.
    \item \(f\) е биекция, ако е инекция и сюрекция. Още се казва взаимноеднозначно изображиние.
\end{itemize}

\begin{example}
    Сядането на хора в зала е частична функция с домейн хората и кодомейн столовете, ако никой не седи на 
    повече от един стол, възможно е да има правостоящи.
    
    Ако няма правостоящи, сядането е функция.

    Ако на никой стол не седи повече от един човек, сядането е инекция.

    Ако няма празни столове, сядането е сюрекция.

    Ако всеки човек седи на отделен стол и няма празни столове, сядането е биекция. Тогава броят на 
    столовете е равен на броя на хората.
\end{example}

Нека \mexpr{X = \{x_1, x_2, ..., x_m\}, Y = \{y_1, y_2, ..., y_n\}} и \mexpr{f : X \to Y}.

Необходима условие \(f\) да е:
\begin{itemize}
    \item инекция е \mexpr{m \le n}
    \item сюрекция е \mexpr{m \ge n}
    \item биекция е \mexpr{m = n}
\end{itemize}

Иначе казано, при:
\begin{itemize}
    \item \mexpr{m > n} няма инекция
    \item \mexpr{m < n} няма сюрекция
    \item \mexpr{m \not = n} няма биекция
\end{itemize}

\begin{definition}
    Нека \mexpr{f : X \to Y} е биекция.
    \bu{Обратната функция} на \(f\) се бележи с \(f^{-1}\). Тя е с домейн \(Y\) и кодомейн \(X\) и се 
    дефинира така: \mexpr{\forall y \in Y : f^{-1}(y) = x}, където \(x\) е уникалният елемент на \(X\), такъв
    че \mexpr{f(x) = y}.
\end{definition}

\subsection{Крайни множества и брой на елементите. Безкрайни изброими множества}
Кардиналност (мощност) на множество \(A\) е броят на елементите му и се бележи с \(|A|\).

\begin{definition} 
    Множество \(A\) е \bu{крайно}, ако:
    \begin{itemize}
        \item \mexpr{A = \emptyset}, тогава \mexpr{|A| = 0}
        \item или съществува \mexpr{n \in \mathbb{N}^+}, такова че съществува биекция
        \mexpr{f : A \to \{1, 2, ..., n\}}. Тогава \mexpr{|A| = n}.
    \end{itemize}
\end{definition}

\begin{definition}
    Множество е \bu{безкрайно}, ако не е крайно.
\end{definition}

\begin{definition}
    Множество е \bu{изброимо безкрайно}, ако е равномощно на \(\mathbb{N}\).
\end{definition}

\begin{definition}
    Множество е \bu{изброимо}, ако е крайно или изброимо безкрайно.
\end{definition}

\begin{definition}
    Множество е \bu{неизброимо}, ако не е изброимо.
\end{definition}

\subsection{Теорема за съществуване на неизброимо (безкрайно) множество}
\begin{theorem}
    Не съществува биекция \mexpr{f:2^\mathbb{N} \to \mathbb{N}}, т.е. \mexpr{2^\mathbb{N}} е неизброимо.
\end{theorem}
\begin{proof}
    Да допуснем, че \mexpr{2^\mathbb{N}} е изброимо, т.е. съществува биекция 
    \mexpr{h:2^\mathbb{N} \to \mathbb{N}}.
    Ще опровергаем, че характеристичните редици на \(\mathbb{N}\) могат да бъдат изброени.

    Допускаме изброяване на характеристичните редици: \(A_0, A_1, ...\), като всяка характеристична 
    редица се появява точно веднъж. \\
    Нека \mexpr{A_0 = (a_{0, 0}, a_{0, 1}, ...)}, \mexpr{A_1 = (a_{1, 0}, a_{1, 1}, ...)} и т.н. \\
    Представяме си ги написани в безкрайна колона: \\
    \mexpr{A_0 = (a_{0, 0}, a_{0, 1}, a_{0, 2}, a_{0, 3}, ...)} \\
    \mexpr{A_1 = (a_{1, 0}, a_{1, 1}, a_{1, 2}, a_{1, 3}, ...)} \\
    \mexpr{A_2 = (a_{2, 0}, a_{2, 1}, a_{2, 2}, a_{2, 3}, ...)} \\
    \mexpr{A_3 = (a_{3, 0}, a_{3, 1}, a_{3, 2}, a_{3, 3}, ...)} \\
    ... \\
    Разглеждаме главния диагонал: редицата \mexpr{X = (a_{0, 0}, a_{1, 1}, a_{2, 2}, ...)}. \\
    Образуваме нейната "побитова инверсия", т.е. редицата \\
    \mexpr{\overline{X} = (\overline{a_{0, 0}}, \overline{a_{1, 1}}, \overline{a_{2, 2}}, ...)}. \\
    За всеки \mexpr{i, j: \overline{a_{i, j}} = 0}, ако \mexpr{a_{i, j} = 1} и обратно.
    Щом всяка булева числова редица се среща в изброяването (колоната), трябва и \(\overline{X}\) да 
    се среща. Но \(\overline{X}\) не може да е \(A_0\), защото се различават в поне една позиция - нулевата.
    Ако \mexpr{a_{0, 0} = 0}, то \mexpr{\overline{a_{0, 0}} = 1} и ако \mexpr{a_{0, 0} = 1}, 
    то \mexpr{\overline{a_{0, 0}} = 0}. \\
    Аналогично, \(\overline{X}\) не може да е \(A_1\), защото се различават в първата позиция, 
    \(\overline{X}\) не може да е \(A_2\), защото се различават във втората позиция и т.н. \\
    Тогава \(\overline{X}\) не се среща в колоната, иначе казано, подмножеството \(B\) на \(\mathbb{N}\),
    съответстващо на \(\overline{X}\), няма образ в хипотетичната биекция \mexpr{h:2^\mathbb{N} \to \mathbb{N}}.
\end{proof}

Алтернативно доказателство:
\begin{theorem}
    За всяко множество \(A\), не съществува сюрекция \mexpr{g: A \to 2^A}.
\end{theorem}
\begin{proof}
    Да допуснем противното - съществува множество \(A\), такова че съществува сюрекция \mexpr{g: A \to 2^A}. \\
    Разглеждаме множеството \mexpr{S = \{a \in A | a \not \in g(a)\}} (1) \\
    Но \mexpr{S \in 2^A} и \(g\) е сюрекция, следователно \mexpr{\exists x \in A : g(x) = S}.
    Тогава:
    \begin{itemize}
        \item ако \mexpr{x \in S}, то \mexpr{x \not \in S} съгласно (1) \(\implies\) противоречие
        \item ако \mexpr{x \not \in S}, то \mexpr{x \in S} съгласно (1) \(\implies\) противоречие
    \end{itemize}
\end{proof}

\subsection{Бележки}
\begin{definition}
%    Нека \mexpr{f : X \to Y} и \mexpr{X^' \subseteq X}.
%    \bu{Рестрикция} на \(f\) върху \(X^'\) е \mexpr{f^' : X^' \to Y}, където 
%    \mexpr{\forallx \in X^': f^'(x) = f(x)}. Бележим рестрикцията на \(f\) с \mexpr{f_{|_{X^'}}}.
\end{definition}

\begin{definition}
    Характеристична редица
\end{definition}