\section{Тема 7}

\subsection{Теорема за мощността на декартовото произведение на две изброимо безкрайни множества}
\begin{theorem}
    Нека \(A\) и \(B\) са множества, \mexpr{|A| = n, |B| = m}. Тогава \mexpr{|A \times B| = n*m}.    
\end{theorem}
\begin{proof}
    Ще докажем твърдението с индукция по \(n\) - броя на елементите на множеството \(A\). \\
    \bu{База}: \mexpr{n = 0} \\
    Тогава \(A = \emptyset\) и \mexpr{\emptyset \times B = \emptyset}. \\
    Следователно \mexpr{|\emptyset \times B| = 0 = 0 * m}. \\ \\
    \bu{Индукционно предположение (ИП)}: нека твърдението е в сила за множество с \(n\) елемента, т.е.
    \mexpr{A = \{a_1, ..., a_n\}} и \mexpr{B = \{b_1, ..., b_m\}}, където \(m\) е произволно. Тогава
    \mexpr{|A \times B| = n * m}. \\ \\
    \bu{Индукционна стъпка}: ще докажем твърдението за множество с \(n + 1\) елемента, т.е. нека 
    \mexpr{A = \{a_1, ..., a_n, a_{n + 1}\}} и \mexpr{B = \{b_1, ..., b_m\}}, където \(m\) е произволно. \\ \\
    Тогава \mexpr{A \times B = \{a_1, a_2, ..., a_n, a_{n + 1}\} \times \{b_1, b_2, ..., b_m\} = 
    \{a_1, a_2, ..., a_n\} \times \{b_1, b_2, ..., b_m\} \cup \{a_{n + 1}\} \times \{b_1, b_2, ..., b_m\}}. \\
    Съгласно ИП, \mexpr{|\{a_1, a_2, ..., a_n\} \times \{b_1, b_2, ..., b_m\}| = n * m} \\
    Броят на елементите в множеството \mexpr{\{a_{n + 1}\} \times \{b_1, b_2, ..., b_m\} = \\
    \{(a_{n + 1}, b_1), (a_{n + 1}, b_2), ..., (a_{n + 1}, b_m)\}} е \(m\). \\
    Множествата \mexpr{\{a_1, a_2, ..., a_n\} \times \{b_1, b_2, ..., b_m\}} и \mexpr{\{a_{n + 1}\} 
    \times \{b_1, b_2, ..., b_m\}} са непресичащи се, защото елементите на първото множество са наредени 
    двойки с първи елемент различен от \(a_{n + 1}\), а елементите на второто множество са наредени 
    двойки с първи елемент \(a_{n + 1}\). \\ \\
    Тогава \mexpr{|A \times B| = n * m + m = (n + 1) * m}.
\end{proof}

\subsection{Теорема за за мощността на степенното множество на изброимо безкрайно множество}
\begin{theorem}
    Нека \(A\) е множество с \(n\) на брой елемента. Тогава \(2^A\) има \(2^n\) на брой елемента, т.е. 
    \mexpr{|2^A| = 2^{|A|} = 2^n}.
\end{theorem}
\begin{proof}
    Ще докажем твърдението с индукция по \(n\). \\
    \bu{База}: \mexpr{n = 0} \\
    Тогава \mexpr{A = \emptyset} и \mexpr{2^\emptyset = \{\emptyset\}}.
    Следователно \mexpr{|2^A| = |2^\emptyset| = 2^{|\emptyset|} = 2^0 = 1}. \\ \\
    \bu{Индукционно предположение (ИП)}: нека твърдението е изпълнено за множество с \(n\) елемента, т.е. 
    \mexpr{|A| = n}. \\ \\
    \bu{Индукционна стъпка}: ще докажем твърдението за множество с \(n + 1\) елемента, т.е. 
    \mexpr{|A| = n + 1}. \\ \\
    Нека \mexpr{x \in A}. Ще разделим \(2^A\) на две непресичащи се множества:
    \mexpr{U_0 = \{B | B \subseteq A \land x \not \in B\}} и \mexpr{U_1 = \{B | B \subseteq A \land x \in B\}}. \\
    Така \mexpr{2^A = U_0 \cup U_1} и съгласно принципа за събирането \mexpr{|2^A| = |U_0| + |U_1|}. \\ \\
    Имаме, че \mexpr{B \in U_0 \iff B \subseteq A \setminus \{x\}}. Следователно \mexpr{U_0 = 2^{A \setminus \{x\}}}, 
    като \(A \setminus \{x\}\) е множество с \(n\) елемента и съгласно ИП \mexpr{|U_0| = 2^n}. \\ \\
    Всеки елемент \mexpr{B \in U_1} може да се представи като \mexpr{B = \{x\} \cup C}, където \mexpr{C \in U_0}.
    Следователно броят на елементите на \(U_1\) е равен на броя на елементите на \(U_0\). \\ \\
    Така \mexpr{|2^A| = |U_0| + |U_1| = 2^n + 2^n = 2^{n + 1}}.
\end{proof}

\subsection{Теорема за за съществуване на минимален и максимален елемент във всяка крайна частична наредба}
\begin{theorem}
    Нека \(A\) е крайно множество и нека \(R \subseteq A^2\) е частична наредба. Тогава \(R\) има поне един 
    минимален и поне един максимален елемент.
\end{theorem}
\begin{proof}
    Да допуснем, че съществува крайно \(A\) и поне една частична наредба \(R\) над \(A\), такава че \(R\) няма
    минимален елемент. \\
    Избираме произволно \(a \in A\). По допускане \(a\) не е минимален, така че 
    \mexpr{\exists b \in A: b \not = a \land bRa}. \\
    По допускане \(b\) не е минимален, така че \mexpr{\exists c \in A: c \not = b \land cRb} и т.н. \\
    В общия случай, веригата \(p\), завършваща на \(a\), изглежда така: \\
    \mexpr{p = z, ..., \underbrace{x, ..., x}_\text{контур}, ..., c, b, a}. \\
    Може \(x\) да е \(a\), или \(b\), или \(c\), или \(z\). \\
    Тогава в \(R\) има контур, което противоречи на това, че в частичните наредби няма контури. \\
    Следователно противоречие с допускането.
\end{proof}

\subsection{Теорема за за разширяване (влагане) на крайна частична наредба до пълна}
\begin{theorem}
    Нека \(A\) е крайно множество, \mexpr{|A| = n} и \mexpr{R \subseteq A^2} е частична наредба. Тогава 
    съществува поне едно линейно разширение \(R'\) на \(R\).
\end{theorem}
\begin{proof}
    (чрез алгоритъма "Топологично сортиране") \\
    Ще построим масив \mexpr{B[1, ..., n]}, в който ще разположим елементите на \(A\). Разполагането на 
    елементите на множество в масив задава еднозначно линейна наредба. \\
    Формално, \(B\) и \(R'\) се съвършено различни обекти - най-малкото 
    \mexpr{|R'| = \frac{n*(n + 1)}{2}}. Но \(R'\) може да бъде конструирана лесно от \(B\). \\

    Вход: крайно множество \(A, |A| = n\), частична наредба \(R \subseteq A^2\) \\
    Изход: масив \(B\), реализиращ линейно разширение \(R'\) на \(R\)
    \begin{algorithmic}[1]
        \State $i \gets 1$ \\
        избираме произволно \mexpr{a \in A}, който е минимален елемент на \(R\) \\
        \mexpr{B[i] \gets a}, изтриваме \(a\) от \(A\) и от \(R\), правим \(i++\) \\
        ако \mexpr{i = n + 1}, върни \(B\), в противен случай иди на 2)
    \end{algorithmic}
\end{proof}

Алгоритъмът е коректен, тъй като в началото има поне един минимален елемент, а при всяко следващо достигане 
на ред 2) пак има поне един минимален елемент, тъй като изтриване на елемент от релацията не може да 
образува цикъл, тя остава частична наредба след всяко изтриване на ред 3).
