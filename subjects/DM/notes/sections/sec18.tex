\section{Тема 18}

\subsection*{Пълни множества БФ. Теорема на Бул.}

\begin{definition}[неформално]
    М-вото \(F \subseteq F_2\) е \underline{пълно}, ако всяка булева ф-я \(f \in F_2\) може да бъде 
    представена като композиция на ф-иите от \(F\).
\end{definition}

\begin{definition}[формално]
    Нека \([F]\) означава затварянето на \(F\) спрямо композиция, което може да се дефинира чрез следната 
    индуктивна дефиниция - \([F]\) е най-малкото м-во, такова че:
    \begin{itemize}
        \item \([F]\) съдържа всички ф-ии от \(F\)
        \item Нека \(f\) е ф-я, която се съдържа в \([F]\) и има \(n \ge 1\) променливи. Нека 
        идентифицираме някои от применливите на \(f\). Получената ф-я се съдържа в \([F]\).
        \item Нека \(f\) и \(g\) са произволни ф-ии от \([F]\) и \(f\) има \(n \ge 1\) променливи. Тогава 
        композицията на \(g\) на мястото на \(i\)-тата променлива на \(f\) също се съдържа в \([F]\) за
        \mexpr{1 \le i \le n}.
    \end{itemize}
    Така \(F\) е \underline{пълно м-во}, ако \([F] = F_2\).
\end{definition}

\begin{theorem}[на Бул]
    М-вото от 3-те булеви ф-ии конюнкция, дизюнкция и отрицание \(F = \{\vee, \land, \lnot\) е пълно.
\end{theorem}

\subsection*{Пълнота на множество БФ чрез свеждане до известно пълно множество.}

\begin{lemma}
    Нека \mexpr{F, G \subseteq F_2} са такива, че \(F \subseteq [G]\). Тогава \mexpr{[F] \subseteq [G]} и 
    \(G\) е пълно м-во.
\end{lemma}

\begin{proof}
    Нека \mexpr{[F] = \cup_{n \in \mathbb{N}} F_n}, където \mexpr{F_0 = F \cup \{I_k^n | 1 \le k \le n\}} и 
    \mexpr{F_{n + 1} = F_n \cup \{h | \exists f, g_1, ..., g_k \in F_n(h(x_1, ..., x_n) = f(g_1(x1_, ..., x_n), 
    ..., g_k(x_1, ..., x_n)))\}}. \\
    Нека \mexpr{G = \cup_{n \in \mathbb{N}} G_n}, където \mexpr{G_0 = G \cup \{I_k^n | 1 \le k \le n\}} и 
    \mexpr{G_{n + 1} = G_n \cup \{h | \exists f, g_1, ..., g_k \in G_n(h(x_1, ..., x_n)) = 
    f(g_1(x_1, ..., x_n), ..., g_k(x_1, ..., x_n))\}}. \\
    Ще докажем, че \mexpr{\forall n : F_n \subseteq [G]}. \\
    По условие имаме, че \(F \subseteq [G]\), тогава \mexpr{\{I_k^n | 1 \le k \le n\} \in G_0 \subseteq [G]}. \\
    Следователно \(F_0 \subseteq [G]\). \\
    Нека \(F_n \subseteq [G]\) и \mexpr{h(x_1, ..., x_n) = f(g_1(x_1, ..., x_n), ..., g_k(x_1, ..., x_n))}, 
    където \mexpr{f, g_1, ..., g_k \in F_n}. \\
    Съгласно И.П. \mexpr{f \in [G] = \cup_{n \in \mathbb{N} G_n}}. Следователно съществува \(m_0\), такова 
    че \(f \in G_{m_0}\). Аналогично, \mexpr{g_1, ..., g_k \in [G]}, следователно съществуват 
    \mexpr{m_1, ..., m_k}, такива че \mexpr{g_i \in G_{m_i}} за \(i = 1, ..., k\). \\
    Нека \mexpr{m = max(m_0, ..., m_k)}. Тогава \mexpr{f, g_1, ..., g_k \in G_m} и \(h \in G_{m + 1}\). \\
    Следователно \(h \in [G]\) и \(F_{n + 1} \subseteq [G]\).
\end{proof}

\subsection*{Литерали, конюнктивни и дизюнктивни клаузи, съвършена ДНФ.}
Нека са фиксирани краен брой булеви променливи \(x_1, ..., x_n, n \ge 1\).

\begin{definition}
    \underline{Литерал} ще наричаме всяко име на променлива (без или с черта отгоре - отрицание). Тези 
    без черта отгоре се наричат положителни, а другите - отрицателни. 
\end{definition}

Литералите са формули и като такива са качествено различни от самите променливи, понеже формулите са 
понятия от синтактичното ниво, а променливите са от по-високото - семантичното ниво.

\begin{example}
    Положителни литерали са \(x_1, x_2\) и т.н., а отрицателни - \(\overline{x_1}, \overline{x_3}\) и т.н.
\end{example}

\begin{definition}
    \underline{Конюктивна клауза} е всяка непразна формула, която се състои от конкатенация на литерали, 
    такива че всяко име на променлива се появява най-много веднъж (без значение като положителна или 
    отрицателна).
\end{definition}

\begin{example}
    Ако \(x_1, ..., x_6\) са променливи, то конюнктивни клаузи са 
    \begin{align*}
        x_1x_3x_5, \\
        x_1\overline{x_6} \\
        \overline{x_2}\overline{x_4}\overline{x_6}
    \end{align*} и т.н.
\end{example}

\begin{definition}
    \underline{Дизюнктивна клауза} е всяка непразна формула, която се състои от литерали, свързани 
    чрез дизюнкция, такива че всяко име на променлива се появява най-много веднъж (без значение 
    като положителна или отрицателна).
\end{definition}

\begin{example}
    Примери са
    \begin{align*}
        x_1 \vee x_2 \\
        x_1 \vee \overline{x_4} \\
        x_3
    \end{align*} и т.н.
\end{example}

\begin{definition}
    \underline{Пълна конюктивна клауза} е конюктивна клауза, която съдържа точно \(n\) литерала. Тя е 
    непразна формула, която се състои от конкатенация на литерали, такива че всяко име на променлива се 
    среща точно веднъж (без значение като положителна или отрицателна).
\end{definition}

\begin{example}
    Ако \(x_1, ..., x_6\) са променливи, то конюнктивни клаузи са 
    \begin{align*}
        x_1x_2x_3x_4x_5x_6 \\
        x_1x_2x_3x_4x_5\overline{x_6}
    \end{align*} и т.н.
\end{example}

\begin{definition}
    \underline{Пълна дизюнктивна клауза} е дизюнктивна клауза, която съдържа точно \(n\) литерала. Тя е 
    непразна формула, която се състои литерали, свързани чрез дизюнкция, такива че всяко име на променлива 
    се среща точно веднъж (без значение като положителна или отрицателна).
\end{definition}

\begin{example}
    Примери са
    \begin{align*}
        x_1 \vee x_2 \vee \overline{x_3} \vee x_4 \vee x_5 \vee \overline{x_6} \\
        x_1 \vee x_2 \vee x_3 \vee x_4 \vee x_5 \vee x_6
    \end{align*}
\end{example}

\begin{definition}
    \underline{Конюктивна нормална форма (КНФ)} е формула, която се състои от една или повече различни 
    конюнктивни клаузи, свързани чрез конюнкция. (Ако дизюнктивните клаузи са повече от една, то всяка 
    от тях се огражда в скоби).
\end{definition}

\begin{example}
    Примери са
    \begin{align*}
        x_1 \vee x_2 \vee x_6 \\
        (x_1 \vee \overline{x_5})(\overline{x_6} \vee x_1)
    \end{align*} и т.н.
\end{example}

\begin{definition}
    \underline{Дизюнктивна нормална форма (ДНФ)} е формула, която се състои от една или повече различни 
    конюнктивни клаузи, свързани чрез дизюнкция.
\end{definition}

\begin{example}
    Примери са
    \begin{align*}
        x_2\overline{x_3}x_4 \\
        x_1x_4 \vee x_2\overline{x_5} \vee \overline{x_3}
    \end{align*} и т.н.
\end{example}

\begin{definition}
    \underline{Съвършена конюктивна нормална форма (СъвКНФ)} е конюктивна нормална форма, в която 
    участват само пълни дизюнктивни клаузи.
\end{definition}

\begin{example}
    Примери са 
    \begin{align*}
        (x_1 \vee x_2 \vee x_3 \vee \overline{x_4} \vee x_5 \vee \overline{x_6})(\overline{x_1} \vee x_2 
        \vee x_3 \vee x_4 \vee \overline{x_5} \vee x_6)
    \end{align*} и т.н.
\end{example}

\begin{definition}
    \underline{Съвършена дизюнктивна нормална форма (СъвДНФ)} е дизюнктивна нормална форма, в която 
    участват само пълни конюнктивни клаузи.
\end{definition}

\begin{example}
    Примери са 
    \begin{align*}
        x_1x_2x_3x_4x_5x_6 \vee \overline{x_1}x_2\overline{x_3}x_4\overline{x_5}x_6 \\
        \overline{x_1}x_2\overline{x_3}x_4\overline{x_5}x_6 \vee x_1\overline{x_2}x_3\overline{x_4}x_5\overline{x_6}
    \end{align*} и т.н.
\end{example}

\begin{note}
    Семантиката на литералите, конюнктивните клаузи, ДНФ и КНФ е следната:
    \begin{itemize}
        \item семантиката на всеки положителен литерал \(x_i\) е ф-ята идентитет - \(x_i\), а на всеки 
        отрицателен литерал \(\overline{x_i}\) e ф-ята отрицание на \(x_i\).
        \item семантиката на всяка конюктивна клауза \mexpr{\lambda_1\lambda_2 ... \lambda_k}, където 
        \(\lambda_i\) са литерали за \mexpr{1 \le i \le k}, е композицията на \mexpr{f_{con} (f_1, f_2, ..., f_k)}, 
        където \(f_{con}\) е обобщената конюнкция на \(k\) променливи, а \(f_i\) е семантиката на \(\lambda_i\) за 
        \(1 \le i \le k\).
        \item семантиката на всяка дизюнктивна клауза \mexpr{\lambda_1 \vee \lambda_2 \vee ... \vee \lambda_k}, 
        където \(\lambda_i\) са литерали за \mexpr{1 \le i \le k}, е композицията на 
        \mexpr{f_{dis} (f_1, f_2, ..., f_k)}, където \(f_{dis}\) е обобщената дизюнкция на \(k\) променливи, 
        а \(f_i\) е семантиката на \(\lambda_i\) за \(1 \le i \le k\).
        \item семантиката на всяка КНФ \mexpr{(\phi_1)(\phi_2) ... (\phi_k)}, където 
        \(\phi_i\) е дизюнктивна клауза за \(1 \le i \le k\), е \mexpr{f_{con} (f_1, f_2, ..., f_k)}, където 
        \(f_{con}\) е обобщената конюнкция на \(k\) променливи, а \(f_i\) е семантиката на \(\phi_i\) за 
        \(1 \le i \le k\).
        \item семантиката на всяка ДНФ \mexpr{\phi_1 \vee \phi_2 \vee ... \vee \phi_k}, където 
        \(\phi_i\) е конюктивна клауза за \(1 \le i \le k\), е \mexpr{f_{dis} (f_1, f_2, ..., f_k)}, където 
        \(f_{dis}\) е обобщената дизюнкция на \(k\) променливи, а \(f_i\) е семантиката на \(\phi_i\) за 
        \(1 \le i \le k\).
    \end{itemize}
\end{note}

\subsection*{Полиноми на Жегалкин – съществуване, единственост и алгоритми за получаване.}

\begin{definition}
    Ф-я от вида \mexpr{f(x_1, ..., x_n) = a_0 \oplus \bigoplus_{1 \le i \le n} a_ix_i \oplus 
    \bigoplus_{1 \le i < j \le n} a_{ij}x_ix_j \oplus ... \oplus a_{12...n}x_1x_2...x_n}, където 
    \(a_i \in \{0, 1\}\), наричаме \underline{полином на Жегалкин}.
\end{definition}

\begin{tabular}{| c | c |}
    \hline
    0 & 0 \\
    \hline
    1 & 1 \\
    \hline
    \(x\) & \(x\) (или \(x \oplus 1\)) \\
    \hline
    \(xy\) & \(xy\) \\
    \hline
    \(x \vee y\) & \(xy \oplus x \oplus y\) \\
    \hline
    \(x \to y\) & \(xy \oplus x \oplus 1\) \\
    \hline
    \(x \iff y\) & \(x \oplus y \oplus 1\) \\
    \hline
    \(x \oplus y\) & \(x \oplus y\) \\
    \hline
    \(x | y\) & \(xy \oplus 1\) \\
    \hline
    \(x \downarrow y\) & \(xy \oplus x \oplus y \oplus 1\) \\
    \hline
\end{tabular}

\begin{theorem}[за съществуване и единственост на полинома на Жегалкин]
    Всяка булева ф-я може да се представи по единствен начин чрез полинома на Жегалкин (т.е. има 
    единствен полином на Жегалкин).
\end{theorem}

\begin{proof}
    \(\newline I)\text{ Комбинаторно д-во}\) \\
    Нека разгледаме общия вид на полинома на Жегалкин на \(n\) променливи:
    \begin{equation*}
        a_0 \oplus \bigoplus_{1 \le i \le n} a_ix_i \oplus \bigoplus_{1 \le i < j \le n} a_{ij}x_ix_j 
        \oplus ... \oplus a_{12...n}x_1x_2...x_n
    \end{equation*}
    Различните ф-ии на \(n\) аргумента имат различни представяния чрез полинома на Жегалкин, като 
    различните полиноми се получават от различните стойности на коефициентите, чиито брой е:
    \begin{itemize}
        \item 1 - свободния член
        \item \(n\) - коефициента пред линейните събираеми
        \item \(\binom{n}{3}\) - по един коефициент за всяка тройка променливи
        \item \dots
        \item \(\binom{n}{k}\) - по един за всяка \(k\)-торка променливи
        \item \dots
        \item \(\binom{n}{n}\) = 1 - пред \(x_1x_2...x_n\)
    \end{itemize}
    Общо \(\sum_{i = 0}^k \binom{n}{k} = 2^n\), съгласно теоремата на Нютон. Всеки коефициент може да има 
    стойност 0 или 1. Общо различните полиноми на Жегалкин са \(2^{2^n}\), колкото и булевите ф-ии. 
    Следователно всяка булева ф-я има единствен полином на Жегалкин.
    \(\newline II)\)Да допуснем, че \(f\) има две различни представяния:
    \begin{align*}
        f(x_1, ..., x_n) &= a_0 \oplus \bigoplus_{1 \le i \le n} a_ix_i \oplus \bigoplus_{1 \le i < j \le n}
        a_{ij}x_ix_j \oplus ... \oplus a_{12...n}x_1x_2...x_n \\
                        &= b_0 \oplus \bigoplus_{1 \le i \le n} b_ix_i \oplus \bigoplus_{1 \le i < j \le n}
        b_{ij}x_ix_j \oplus ... \oplus b_{12...n}x_1x_2...x_n
    \end{align*}
    Нека \(a_{i_1...i_k}\) и \(b_{i_1...i_k}\) са първата двойка различни коефициенти. Заместваме в двете 
    представяния с вектора \(c_1, ..., c_n\), където
    \begin{equation*}
        c_j =
          \begin{cases}
            1 & \text{, ако } j \in \{i_1, ..., i_k\} \\
            0 & \text{, ако } j \not \in \{i_1, ..., i_k\}
          \end{cases}       
    \end{equation*}
    Събираемите до това с индекс \({i_1...i_k}\) и в двете представяния са еднакви, нека те са с обща 
    стойност \(C\). \\
    Всички събираеми след това с индекс \({i_1...i_k}\) се нулират, защото в тях участва като множител 
    \(c_j\) за някое \mexpr{j \not \in \{i_1, ..., i_k\}}. \\
    Така \mexpr{f(c_1, ..., c_n) = C \oplus a_{i_1...i_k} = C \oplus b_{i_1...i_k}}, което противоречи на 
    допускането, че \mexpr{a_{i_1...i_k} \not = b_{i_1...i_k}}.
\end{proof}

\begin{alg}[за получаване на полинома на Жегалкин]
    \(\newline I)\) от СъвДНФ \\
    Негираните променливи се заместват съответно с тяхната ненегирана форма, като всяка от променливите се 
    събира по модул 2 с единица, за да се запази смисълът на формулата (т.е. използва се еквивалентност 
    на негация за \mexpr{\{\tilde{1}, \tilde{2}, \land, \oplus\}}). Тъй като в СъвДНФ има само един 
    елемент със стойност единица е допустимо замистването на дизюнкцията със сума по модул 2.
    \begin{example}
        \begin{align*}
            \overline{x}\overline{y} \vee \overline{x}y \vee xy &= (1 \oplus x)(1 \oplus y) \vee 
            ((1 \oplus x)y) \vee xy \\
            &= (1 \oplus x)(1 \oplus y) \oplus (1 \oplus x) y \oplus xy \\
            &= 1 \oplus x \oplus y \oplus y \oplus xy \\
            &= 1 \oplus x \oplus xy
        \end{align*}
    \end{example}
    \(\newline II)\) Чрез заместване \\
    Замествайки всички възможни вектори за променливите в общата формула на полинома на Жегалкин се 
    получават коефициентите пред променливите и след това се заместват в общата формула.
    \begin{example}
        \begin{align*}
        f(x, y) = a_0 + a_1x + a_2y + a_3xy \\
        \systeme*{
            f(0\text{,} 0) = 1, 
        f(0\text{,} 0) = a_0 \oplus a_1.0 \oplus a_2.0 \oplus a_3.0.0 = a_0} \\
        \implies a_0 = 1 \\
        \systeme*{
            f(0\text{,} 1) = 1, 
        f(0\text{,} 1) = 1 \oplus a_1.0 \oplus a_1 \oplus a_3.1.0 = 1 \oplus a_2 = a_2} \\
        \implies a_2 = 0 \\
        \systeme*{
            f(1\text{,} 0) = 0, 
            f(1\text{,} 0) = 1 \oplus a_1.1 \oplus a_2.0 \oplus a_3.1.0 = 1 \oplus a_1} \\
        \implies a_1 = 1 \\
        \systeme*{
            f(1\text{,} 1) = 1, 
            f(1\text{,} 1) = 1 \oplus 1.1 \oplus 0.1 \oplus a_3.1.1 = 1 \oplus 1 \oplus a_3 = 0 \oplus a_3} \\
        \implies a_3 = 1
        \end{align*}
        Следователно полинома на Жегалкин е \(1 \oplus x \oplus xy\).
    \end{example}
    \(\newline III)\) Чрез еквивалентни преобразувания \\
    Дадените операции се заместват с техните съответсващи от \mexpr{\{\tilde{0}, \tilde{1}, \land, \oplus\}}.
    \begin{example}
        \begin{align*}
            x \to y &\equiv \text{// св-во на импликацията} \\
            \overline{x} \vee y &\equiv \text{// закон за двойното отрицание} \\
            \overline{\overline{\overline{x} \vee y}} &\equiv \text{// закон на де Морган} \\
            \overline{\overline{\overline{x}}\overline{y}} &\equiv \text{// закон за двойното отрицание} \\
            \overline{x\overline{y}} &\equiv \overline{x (1 \oplus y)} \equiv 1 \oplus (x (1 \oplus y)) \\ 
            &= 1 \oplus x \oplus xy
        \end{align*}
    \end{example}
\end{alg}