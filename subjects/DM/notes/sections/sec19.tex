\section{Тема 19}

% TODO: find information
\subsection{Функционални елементи. Дефиниция на схема от ФЕ. Построяване на схема от ФЕ от СъвДНФ.}

Нека са дадени устройства, всяко от които има 0 или повече входове и точно един изход. На входовете се
подават булеви стойности. За всяка комбинация от булеви стойности на входовете, на изхода "излиза" булева
стойност, еднозначно определена от това, което е подадено от входовете.

\begin{definition}
	Ако броят на входовете е \(n\), то всяко такова устройство реализира някаква булева функция на
	\(n\) променливи, като всеки вход се смята за една булева променлива. Всяко такова устройство се
	нарича \textbf{функционален елемент}.
\end{definition}

\subsection{Пример с двоичен суматор}

\includepdf[pages=-]{./sections/binary-addition-circuit.pdf}
