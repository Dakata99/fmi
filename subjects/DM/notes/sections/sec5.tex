\section{Тема 5} % TODO:

\subsection{Частични наредби}
\begin{definition}
    Релация е релация на частична наредба \totw е рефлексивна, антисиметрична и транзитивна.
\end{definition}

\begin{example}
    \(=, \le, \ge\) над \(\mathbb{R}^2\)
\end{example}

При частичните наредби може (но не непременно) да има несравними елементи (примерно нарадените двойки 
\((4, 5)\) и \((5, 4)\)). Елементи \(a\) и \(b\) са несравними, ако \mexpr{\lnot aRb} и \mexpr{\lnot bRa}.

\subsection{Линейни наредби}
\begin{definition}
    Релация е релация на линейна наредба \totw е рефлексивна, силно антисиметрична и транзитивна.
\end{definition}

Не може да има несравними елементи заради силната антисиметричност.
Ако \(R \subseteq A^2\) е линейна наредба и \mexpr{A = \{a_1, ..., a_n\}}, то \(R\) има точно 
\mexpr{\frac{n(n + 1)}{2}} елемента.

ЛН-и са частен случай на ЧН-и.

\subsection{Вериги и контури в релации}
Ако \(R \subseteq A^2\) е произволна релация и \mexpr{A = \{a_1, ..., a_n\}}, \bu{верига} в \(R\) е всяка
редица \mexpr{a_{i_0}, ..., a_{i_k}}, където \mexpr{i_0, ..., i_k \in \{1, ..., n\}}, ако 
\mexpr{a_{i_j}Ra_{i_{j + 1}}} за \mexpr{0 \le j \le k - 1}. Ограничение за \(k\) няма, тоест \(k \ge 0\). 
Тогава един единствен елемент е верига.

Ако \(a_{i_0} = a_{i_k}\) и \(k > 0\), то веригата е \bu{контур}. Тогава \(k > 0\) налага \(k > 1\). 
Един единствен елемент не е контур.

\begin{example}
    
\end{example}

\subsection{Теорема за контурите}
\begin{theorem}
    Нека \mexpr{R \subseteq A^2} е рефлексивна и транзитивна релация. Тогава \(R\) е частична наредба 
    \totw няма контури.

    \begin{proof}
        $ $\newline
        \(\Rightarrow):\) Нека \(R\) е частична наредба. \\
        Ще докажем, че \(R\) няма контури. \\
        Да допуснем, че \(R\) има контур \mexpr{a_{i_0}, a_{i_1}, ..., a_{i_{k - 1}}, a_{i_k} = a_{i_0}}. \\
        От определението за контур и транзитивността на \(R\) имаме, че: 
        \begin{align*}
            a_{i_0}Ra_{i_1} \land a_{i_1}Ra_{i_2} \implies a_{i_0}Ra_{i_2} \\
            a_{i_0}Ra_{i_2} \land a_{i_2}Ra_{i_3} \implies a_{i_0}Ra_{i_3} \\
            \dots \\ 
            a_{i_0}Ra_{i_{k - 2}} \land a_{i_{k - 2}}Ra_{i_{k - 1}} \implies a_{i_0}Ra_{i_{k - 1}}
        \end{align*}
        Но тогава имаме, че \mexpr{a_{i_0}Ra_{i_{k - 1}}} и \mexpr{a_{i_{k - 1}}Ra_{i_0}} (понеже \mexpr{a_{i_k} = a_{i_0}} от 
        определението за контур). \\
        Следователно \(R\) не е антисиметрична \\ \(\implies\) не е частична наредба \\ \(\implies \) \lightning \\
        
        \(\Leftarrow):\) Нека \(R\) е няма контури. \\
        Ще докажем, че \(R\) е частична наредба. \\
        Допускаме противното - \(R\) не е частична наредба. \\
        В началото сме допуснали, че \(R\) е рефлексивна и транзитивна и това допускане е в сила. \\
        Тогава за да не е частична наредба, тя не трябва да е антисиметрична. \\
        Щом не е антисиметрична, то задължително съществуват \mexpr{a, b \in A}, такива че 
        \mexpr{a \not = b: aRb \land bRa} (не е антисиметрична: \mexpr{\lnot(aRb \to \lnot bRa) \equiv \lnot(\lnot aRb \vee \lnot bRa) \equiv
        aRb \land bRa}). \\
        Но тогава \mexpr{a, b, a} е контур \(\implies\) противоречие с това, 
        че \(R\) няма контури.
    \end{proof}
\end{theorem}

\subsection{Влагане на частична наредба в линейна наредба – дефиниция}
Ако \(R \subseteq A^2\) е частична наредба, \mexpr{R^{'} \subseteq A^2} е линейна наредба и 
\mexpr{R \subseteq R^{'}}, казваме, че \(R\) се влага в \(R^{'}\) (или че \(R^{'}\) е линейно 
разширение на \(R\)).

При \mexpr{A = \{a_1, ..., a_n\}}, броят на линейните разширения варира от 1 (самата R е линейна наредба) 
до \(n!\) (няма несравними елементи в \(R\)).

\subsection{Допълнение}
Нека \mexpr{R \subseteq A^2} е частична наредба. За всеки елемент \(a \in A\) казваме, че \(a\) е 
\bu{минимален} в \(R\), ако \mexpr{\lnot \exists b \in A, b \not = a : bRa \equiv \forall b \in A,
b \not = a: \lnot bRa}.

Аналогично, \(a\) е \bu{максимален} в \(R\), ако \mexpr{\lnot \exists b \in A, b \not = a : aRb \equiv 
\forall b \in A, b \not = a: \lnot aRb}.

Може да има повече от един минимален и повече от един максимален елемент или да няма нито минимален, нито 
максимален.