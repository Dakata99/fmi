\section{Тема 1}

% Съждителна логика – прости съждения, логически съюзи, съставни съждения. Таблици на истинност
\subsection{Съждителна логика - съждения, логически съюзи}

Логиката е науката за правенето на валидни изводи и правилни разсъждения. Дали даден
извод е валиден или не, зависи от формата му.

В съждителната логика не се допускат "междинни възможности" на частична истинност, т.е. дадено съждение
е или истина, или лъжа, друга възможност няма.

\begin{definition}
	\textbf{Просто съждение} е просто разказвателно изречение (още се нарича логическа променлива),
	което е или истина, или лъжа. Въпросителните изречения и възклицанията не са съждения, както и
	разказвателните изречения, за които не можем да твърдим, че са или истина, или лъжа.
\end{definition}

\begin{example}
	"Това изречение е лъжа."
\end{example}

\begin{note}
	Бележим истината с \(T\) (или 1), а лъжата с \(F\) (или 0), където \(T\) и \(F\) са
	\textbf{логически константи}.
\end{note}

\begin{definition}
	\textbf{Съставни съждения} се образуват от прости съждения, други съставни съждения и логически
	константи чрез логически съюзи.
\end{definition}

\paragraph{Видове логически съюзи\\}

Нека \(p\) и \(q\) са съждения.

\begin{enumerate}
	\item Дизюнкция (съответства на "или" от естествения език) \\
	      Бележи се с '\mexpr{\vee}'. Дизюнкцията на \(p\) и \(q\) е \mexpr{p \vee q}.
	      Тя е съставно съждение, което е:
	      \begin{itemize}
		      \item лъжа, ако и \(p\), и \(q\) са лъжа,
		      \item истина, във всеки останал случай.
	      \end{itemize}
	      Тоест дизюнкцията е истина \totw поне едно от участващите съждения е истина.

	      Друг начин да се определи е чрез таблица на истинност, в която на всеки ред на таблицата отговаря
	      точно една възможна комбинация от \(F\) и \(T\) за \(p\) и \(q\). Всяко такова "раздаване" на
	      конкретни стойности (\(F\) или \(T\)) на променливите, се нарича \textbf{валюация}. Броят на
	      валюациите при \(n\) променливи е \(2^n\).

	      \begin{center}
		      \begin{tabular}{ | c | c | c | }
			      \hline
			      \(p\) & \(q\) & \mexpr{p \vee q} \\
			      \hline
			      0     & 0     & 0                \\
			      \hline
			      0     & 1     & 1                \\
			      \hline
			      1     & 0     & 1                \\
			      \hline
			      1     & 1     & 1                \\
			      \hline
		      \end{tabular}
	      \end{center}

	\item Изключващо или \\
	      Бележи се с '\mexpr{\oplus}'. Изключващото или на \(p\) и \(q\) е \mexpr{p \oplus q}, което за да бъде
	      истина, се изисква точно едно от участващите съждения да е истина, а другото лъжа.

	      \begin{center}
		      \begin{tabular}{ | c | c | c | }
			      \hline
			      \(p\) & \(q\) & \mexpr{p \oplus q} \\
			      \hline
			      0     & 0     & 0                  \\
			      \hline
			      0     & 1     & 1                  \\
			      \hline
			      1     & 0     & 1                  \\
			      \hline
			      1     & 1     & 0                  \\
			      \hline
		      \end{tabular}
	      \end{center}

	\item Конюнкция (съответства на 'и' от естествения език) \\
	      Бележи се с '\mexpr{\land}'. Конюнкцията на \(p\) и \(q\) е \mexpr{p \land q}. Тя е съставно
	      съждение, което е:
	      \begin{itemize}
		      \item истина, ако и \(p\), и \(q\) са истина,
		      \item лъжа, във всеки останал случай.
	      \end{itemize}
	      Тоест конюнкцията е истина \totw всички участващи съждения са истина.

	      \begin{center}
		      \begin{tabular}{ | c | c | c | }
			      \hline
			      \(p\) & \(q\) & \mexpr{p \land q} \\
			      \hline
			      0     & 0     & 0                 \\
			      \hline
			      0     & 1     & 0                 \\
			      \hline
			      1     & 0     & 0                 \\
			      \hline
			      1     & 1     & 1                 \\
			      \hline
		      \end{tabular}
	      \end{center}

	\item Импликация (съответства на "ако ..., то ...") \\
	      Бележи се с '\mexpr{\rightarrow}'. Импликацията на \(p\) и \(q\) e
	      \mexpr{p \rightarrow q} (чете се "\(p\) импликация \(q\)"). Тя е съставно съждение, което е:
	      \begin{itemize}
		      \item лъжа, ако \(p\) е истина и \(q\) е лъжа,
		      \item истина, във всеки останал случай.
	      \end{itemize}
	      Съждението \(p\) се нарича антецедент и е свързан с достатъчността, \(q\) се нарича консеквент и е
	      свързан с необходимостта.

	      \begin{center}
		      \begin{tabular}{ | c | c | c | }
			      \hline
			      \(p\) & \(q\) & \mexpr{p \rightarrow q} \\
			      \hline
			      0     & 0     & 1                       \\
			      \hline
			      0     & 1     & 1                       \\
			      \hline
			      1     & 0     & 0                       \\
			      \hline
			      1     & 1     & 1                       \\
			      \hline
		      \end{tabular}
	      \end{center}

	\item Би-импликация \\
	      Бележи се с '\mexpr{\leftrightarrow}'. Би-импликацията на \(p\) и \(q\) е \mexpr{p \leftrightarrow q} (чете се "\(p\) \totw \(q\)").
	      Тя е съставно съждение, което е:
	      \begin{itemize}
		      \item истина, когато \(p\) и \(q\) имат една и съща логическа стойност,
		      \item лъжа, във всеки останал случай.
	      \end{itemize}

	      \begin{center}
		      \begin{tabular}{ | c | c | c | }
			      \hline
			      \(p\) & \(q\) & \mexpr{p \leftrightarrow q} \\
			      \hline
			      0     & 0     & 1                           \\
			      \hline
			      0     & 1     & 0                           \\
			      \hline
			      1     & 0     & 0                           \\
			      \hline
			      1     & 1     & 1                           \\
			      \hline
		      \end{tabular}
	      \end{center}

	\item Отрицание (негация) \\
	      Бележи се с '\mexpr{\lnot}'. Прилага се към едно съждение.
	      Отрицанието на \(p\) е \mexpr{\lnot p}, което е:
	      \begin{itemize}
		      \item лъжа, ако \(p\) e истина,
		      \item истина, ако \(p\) е лъжа.
	      \end{itemize}

	      \begin{center}
		      \begin{tabular}{ | c | c | }
			      \hline
			      \(p\) & \mexpr{\lnot p} \\
			      \hline
			      0     & 1               \\
			      \hline
			      1     & 0               \\
			      \hline
		      \end{tabular}
	      \end{center}

\end{enumerate}

Приоритети на логическите съюзи (в намаляващ порядък):
\begin{enumerate}
	\item отрицание
	\item конюнкция, дизюнкция
	\item импликация, би-импликация
\end{enumerate}

\subsection{Еквивалентност на съставни съждения}

\begin{definition}
	Съставно съждение, чиято стойност е \(T\) за всяка валюация на простите му съждения,
	се нарича \textbf{тавтология}.
\end{definition}

\begin{example}
	\mexpr{p \vee \lnot p}
	\begin{center}
		\begin{tabular}{ | c | c | c | }
			\hline
			\(p\) & \(\lnot p\) & \mexpr{p \vee \lnot p} \\
			\hline
			0     & 1           & 1                      \\
			\hline
			1     & 0           & 1                      \\
			\hline
		\end{tabular}
	\end{center}
\end{example}

\begin{definition}
	Съставно съждение, чиято стойност е \(F\) за всяка валюация на простите му съждения, се нарича
	\textbf{противоречие}.
\end{definition}

\begin{example}
	\mexpr{p \land \lnot p}
	\begin{center}
		\begin{tabular}{ | c | c | c | }
			\hline
			\(p\) & \(\lnot p\) & \mexpr{p \land \lnot p} \\
			\hline
			0     & 1           & 0                       \\
			\hline
			1     & 0           & 0                       \\
			\hline
		\end{tabular}
	\end{center}
\end{example}

\begin{definition}
	Съставно съждение, чиято стойност е \(T\) за поне една валюация и \(F\) за поне една валюация
	на простите му съждения, се нарича \textbf{условност}.
\end{definition}

\begin{example}
	\mexpr{p \rightarrow q}
	\begin{center}
		\begin{tabular}{ | c | c | c | }
			\hline
			\(p\) & \(q\) & \mexpr{p \rightarrow q} \\
			\hline
			0     & 0     & 1                       \\
			\hline
			0     & 1     & 1                       \\
			\hline
			1     & 0     & 0                       \\
			\hline
			1     & 1     & 1                       \\
			\hline
		\end{tabular}
	\end{center}
\end{example}

\begin{definition}
	За всеки две съставни съждения \(s\) и \(t\) казваме, че \(s\) и \(t\) са \textbf{еквивалентни}
	\totw \mexpr{s \leftrightarrow t} е тавтология. Бележим с '\mexpr{s \equiv t}'
	('\mexpr{\equiv}' не е логически съюз, така че \mexpr{s \equiv t} не е съставно съждение).
\end{definition}

% Методи за доказателство на еквивалентност: табличен метод и метод с еквивалентни преобразувания. 
\subsection{Методи за доказателство на еквивалентност}

\begin{example}
	\mexpr{(p \rightarrow q) \land p \equiv p \land q}
	\begin{itemize}
		\item Табличен метод
		      \begin{center}
			      \begin{tabular}{ | c | c | c | c | c | }
				      \hline
				      \(p\) & \(q\) & \mexpr{p \rightarrow q} & \mexpr{(p \rightarrow q) \land q} & \mexpr{p \land q} \\
				      \hline
				      0     & 0     & 1                       & 0                                 & 0                 \\
				      \hline
				      0     & 1     & 1                       & 0                                 & 0                 \\
				      \hline
				      1     & 0     & 0                       & 0                                 & 0                 \\
				      \hline
				      1     & 1     & 1                       & 1                                 & 1                 \\
				      \hline
			      \end{tabular}
		      \end{center}
		\item Метод чрез еквивалентни преобразувания \\
		      \mexpr{(p \rightarrow q) \land p \equiv }           // св-во на импликацията \\
		      \mexpr{(\lnot p \vee q) \land p \equiv}             // дистрибутивност \\
		      \mexpr{(\lnot p \land p ) \vee (q \land p) \equiv}  // св-во на отрицанието \\
		      \mexpr{F \vee (q \land p) \equiv}                   // св-во на константите \\
		      \mexpr{(q \land p) \equiv}                          // комутативност \\
		      \mexpr{p \land q}
	\end{itemize}
\end{example}

\subsection{Основни свойства на логическите съюзи}

Нека \(p\), \(q\) и \(r\) са произволни съждения. Тогава следните свойства са в сила:
\begin{enumerate}
	\item Свойство на константите
	      \begin{itemize}
		      \item \mexpr{p \land T \equiv p}
		      \item \mexpr{p \land F \equiv F}
		      \item \mexpr{p \vee T \equiv T}
		      \item \mexpr{p \vee F \equiv p}
	      \end{itemize}
	\item Свойство на отрицанието
	      \begin{itemize}
		      \item \mexpr{p \vee \lnot p \equiv T}
		      \item \mexpr{p \land \lnot p \equiv F}
	      \end{itemize}
	\item Идемпотентност
	      \begin{itemize}
		      \item \mexpr{p \vee p \equiv p}
		      \item \mexpr{p \land p \equiv p}
	      \end{itemize}
	\item Закон за двойното отрицание
	      \begin{itemize}
		      \item \mexpr{\lnot (\lnot p) \equiv p}
	      \end{itemize}
	\item Комутативност
	      \begin{itemize}
		      \item \mexpr{p \vee q \equiv q \vee p}
		      \item \mexpr{p \land q \equiv q \land p;}
		      \item \mexpr{p \oplus q \equiv q \oplus p}
		      \item \mexpr{p \rightarrow q \not \equiv q \rightarrow p}
		      \item \mexpr{p \iff q \equiv q \iff p}
	      \end{itemize}
	\item Асоциативност
	      \begin{itemize}
		      \item \mexpr{(p \vee q) \vee r \equiv p \vee (q \vee r) \equiv p \vee q \vee r}
		      \item \mexpr{(p \land q) \land r \equiv p \land (q \land r) \equiv p \land q \land r}
	      \end{itemize}
	\item Дистрибутивност
	      \begin{itemize}
		      \item \mexpr{p \vee (q \land r) \equiv (p \vee q) \land (p \vee r)}
		      \item \mexpr{p \land (q \vee r) \equiv (p \land q) \vee (p \land r)}
	      \end{itemize}
	\item Закони на де Морган
	      \begin{itemize}
		      \item \mexpr{\lnot (p \land q) \equiv \lnot p \vee \lnot q}
		      \item \mexpr{\lnot (p \vee q) \equiv \lnot p \land \lnot q}
	      \end{itemize}
	\item Закон за поглъщането
	      \begin{itemize}
		      \item \mexpr{p \vee (p \land q) \equiv p}
		      \item \mexpr{p \land (p \vee q) \equiv p}
	      \end{itemize}
	\item Свойство на импликацията
	      \begin{itemize}
		      \item \mexpr{p \rightarrow q \equiv \lnot p \vee q}
	      \end{itemize}
	\item Свойствона би-импликацията
	      \begin{itemize}
		      \item \mexpr{p \iff q \equiv (p \rightarrow q) \land (q \rightarrow p)}
	      \end{itemize}
\end{enumerate}

% Основи на предикатната логика – дефиниция на предикат, универсален и екзистенциален квантор.
\subsection{Основи на предикатната логика}

\begin{definition}
	\textbf{Едноместен предикат} е съждение, в което има ``празно място``, в което празно място се
	слага обект от някаква предварително зададена област, наречена \textbf{домейн}. За всеки обект от домейна
	предикатът е или истина, или лъжа. Бележим с \(P(x)\) (или други главни латински букви), където \(x\) е
	обект от домейна. Сам по себе си \(P(x)\) не е нито истина, нито лъжа. Истина или лъжа се получава само
	след заместване на \(x\) с някой обект от областта, като има два случая на такова заместване:
	\begin{itemize}
		\item когато има поне един обект, за който предикатът е истина. \linebreak
		      Това бележим с \mexpr{\exists x: P(x)}, където ``\(\exists\)`` e \textbf{екзистенциален квантор}.
		      Ако обектите от областта са краен брой, да речем \linebreak
		      \(a_1, a_2, ..., a_n\), то \(\exists x: P(x)\) има смисъл на
		      \(P(a_1) \vee P(a_2) \vee ... \vee P(a_n)\)
		\item когато за всеки обект предикатът е истина. \linebreak
		      Това бележим с \mexpr{\forall x: P(x)}, където ``\(\forall\)`` e \textbf{универсален квантор}.
		      Ако обектите от областта са краен брой, да речем \(a_1, a_2, ..., a_n\),
		      то \(\forall x: P(x)\) има смисъл на \(P(a_1) \land P(a_2) \land ... \land P(a_n)\)
	\end{itemize}
\end{definition}

\begin{note}
	Когато е използван квантор върху някаква променлива, казваме, че тя е \textbf{свързана}. В противен случай,
	казваме, че тя е \textbf{свободна}.

	Предикат, в който всички променливи са свързани, е съждение.
\end{note}

\begin{example}
	В израза \mexpr{\forall x \underbrace{(P(x) \to Q(x))}_\text{обхват на квантора}}, \(x\) e свързана
	променлива. \\
	В израза \mexpr{\forall x \underbrace{P(x)}_\text{обхват на квантора} \to Q(x)}, \(x\) e свободна променлива
	в \(Q(x) \implies Q(x)\) не е нито истина, нито лъжа.
\end{example}

\begin{example}
	Отрицания на изрази с едноместни предикати:
	\begin{itemize}
		\item \mexpr{\lnot \forall x P(x) \equiv \exists x \lnot P(x)}
		\item \mexpr{\lnot \exists x P(x) \equiv \forall x \lnot P(x)}
	\end{itemize}
\end{example}

\begin{definition}
	Предикатите могат да имат повече от едно "празно място" за попълване, т.е. да са двуместни,
	триместни и т.н.
\end{definition}

\begin{note}
	В израза \mexpr{\forall x \forall y P(x, y)} казваме, че кванторите (които може да са от различен вид) са
	\textbf{вложени}.
\end{note}

Еднотипни квантори могат да бъдат размествани без това да се отразява на истинността, тоест винаги:
\begin{itemize}
	\item \mexpr{\forall x \forall y P(x, y) \equiv \forall y \forall x P(x, y)}
	\item \mexpr{\exists x \exists y P(x, y) \equiv \exists y \exists x P(x, y)}
\end{itemize}

От друга страна, разнотипни квантори не може да бъдат размествани, тоест, в общия случай:
\begin{itemize}
	\item \mexpr{\forall x \exists y P(x, y) \not \equiv \exists y \forall x P(x, y)}
	\item \mexpr{\exists x \forall y P(x, y) \not \equiv \forall x \exists y P(x, y)}
\end{itemize}

\begin{note}
	В израза \mexpr{\forall x P(x, y), y} е свободна променлива \(\implies P(x, y)\) не е нито истина,
	нито лъжа.
\end{note}

\subsection{Свойства на отрицанието в предикатната логика}

\begin{example}
	Негация на израз с много квантори:
	\begin{align*}
		\lnot (\forall \epsilon > 0 \exists \delta > 0 \forall x (0 < |x & - a| < \delta \to |f(x) - L| < \epsilon))
		\\ &\equiv \\
		\exists \epsilon > 0 \forall \delta > 0 \exists x (0 < |x        & - a| < \delta \land |f(x) - L| >= \epsilon)
	\end{align*}
\end{example}
