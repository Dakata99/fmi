\section{Тема 3}

\subsection{Индуктивно дефинирани множества}

\begin{axiom}
    Можем да си я представим като безкрайна процедура, която генерира множество, стартирайки от някаква 
    база и прилагайки итеративно дадени операции.

    Нека е дадена непразно множество \(M_0\), което наричаме \textbf{базово множество}, и 
    непразно множество от операции \(F\), приложими в тази конструкция.

    \begin{itemize}
        \item Включваме елементите на \(M_0\) в \(M\), т.е. \mexpr{M \leftarrow M_0}.
        \item Прилагаме неограничено следното:
            \begin{itemize}
                \item[-] нека \(M^{'}\) е множеството от елемнитите, които се получават при всевъзможните 
                прилагания на операциите от \(F\) върху текущото \(M\)
                \item[-] добавяме \(M^{'}\) към \(M\), т.е. \(M \leftarrow M \cup M^{'}\)
            \end{itemize}
    \end{itemize}

    Така полученото \(M\) е множество. Пишем \mexpr{M = (M_0, F)}.

    Множества, генерирани чрез безкрайната процедура от аксиомата за индукцията, 
    наричаме \textbf{индуктивно генерирани множества}.
\end{axiom}

\begin{example}
Множеството на естествените числа \(\mathbb{N}\) \\
При него \mexpr{M_0 = \{0\}}, а \(F\) съдържа една единствена операция - добавяне на единица.
\end{example}

\subsection{Доказателства по индукция - обикновена, силна, структурна. Примери}

Нека е даден предикат \(P(n)\) и трябва да докажем \mexpr{\forall x \in \mathbb{N} : P(n)}.

\begin{scheme}
Схемата на доказателствата по индукция върху естествените числа е следната:
\begin{enumerate}
    \item[(База)] доказваме \(P(0)\), като просто проверяваме истиността на предиката за \(n = 0\)
    \item[(И.П.)] допускаме \(P(n)\) за произволно \mexpr{n \in \mathbb{N}} и въз основа на това 
    допускане доказваме \(P(n + 1)\) (И.С.)
\end{enumerate}
\end{scheme}

\begin{example} % TODO: example
    
\end{example}

\begin{scheme}
Схемата на доказателствата със силна индукция върху естествените числа е следната:
\begin{enumerate}
    \item[(База)] доказваме \(P(0)\), като просто проверяваме истиността на предиката за \(n = 0\)
    \item[(И.П.)] допускаме, че за произволно \mexpr{n \in \mathbb{N}} са изпълнени \mexpr{P(0), P(1), ..., P(n)}
    \item[(И.С.)] въз основа на тези допускания доказваме \(P(n + 1)\)
\end{enumerate}
\end{scheme}

\begin{example} % TODO: example
    
\end{example}

\begin{scheme}
    В по-общия случай доказваме предикат \(P(x)\), където домейнът е някакво индуктивно дефенирано 
    множество \mexpr{M = (M_0, F)}. Схемата на доказателство е следната:
    \begin{itemize}
        \item[(База)] за всеки елемент \(x\) от \(M_0\) проверяваме истиността на \(P(x)\).
        \item[(И.П.)] допускаме \(P(x)\) за произволно \mexpr{x \in M}
        \item[(И.С.)] въз основа на това допускане доказваме, че за всеки елемент \(y\), който се 
        получава при прилагането на операциите от \(F\) върху текущото \(M\), то \(P(y)\) е вярно
    \end{itemize}
\end{scheme}

\begin{example} % TODO: example
    
\end{example}

\subsection{Наредена двойка. Декартово произведение}

\begin{definition}[Дефиниция на Kuratowski]
    Всяко множество \linebreak \mexpr{\{\{a\}, \{a, b\}\}} наричаме \textbf{наредена двойка} с 
    първи елемент \(a\) и втори елемент \(b\). Бележим "\mexpr{(a, b)}".
\end{definition}

\begin{definition}
    Нека \(A\) и \(B\) са множества. \textbf{Декартовото произведение} на \(A\) и \(B\) е множеството 
    \mexpr{A \times B = \{(a, b) | a \in A \land b \in B\}}.
\end{definition}
