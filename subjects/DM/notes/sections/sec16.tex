\section{Тема 16}

\subsection*{Най-къси пътища в графи.}

\begin{definition}
    Нека \(G = (V, E)\) е свързан граф, а \(w: E \to \mathbb{R}^+\) е тегловна ф-я на ребрата с 
    положителни стойности. \underline{Претеглена дължина} на пътя \mexpr{p = v_{i_0}, v_{i_1}, ..., v_{i_t}} 
    в графа ще наричаме 
    \begin{equation*}
        w(p) = \sum_{j = 0}^{t - 1} w(v_{i_j}, v_{i_{j + 1}})
    \end{equation*}
    Пътят от \(v_{i_0}\) до \(v_{i_t}\) с най-малка претеглена дължина наричаме \underline{най-къс път}
    от \(v_{i_0}\) до \(v_{i_t}\).
\end{definition}

\begin{note}
    За тривиалния път от \(v\) до \(v\) за всяко \(v \in V\) имаме претеглена дължина 0.
\end{note}

\subsection*{Най-къси пътища в тегловни граф.}

\begin{theorem}
    Нека \(G = (V, E)\) е свързан граф с тегловна ф-я \(w(e) = 1, \forall e \in E\) и \(D\) е покриващо 
    дърво на \(G\) с корен \(v_0\), построено в ширина. Пътищата в \(D\) от корена до останалите върхове 
    на \(G\) са най-къси пътища от върха \(v_0\) до тези върхове.
\end{theorem}

\begin{proof}
    С индукция по нивата \mexpr{L_i, i = 0, 1, ..., t} на построеното в ширина дърво ще докажем, че дължината 
    на минималния път от корена \(v_0\) до всеки връх от ниво \(L_i\) е точно \(i\). \\
    \bu{База:} Дължината на най-късия път от \(v_0\) до \(v_0\) е 0 и тъй като той е единствен връх 
    в \(L_0\), то твърдението е в сила за \(L_0\). \\
    \bu{Индукционно предположение:} Да допуснем верността за нивата \\ \mexpr{L_0, L_1, ..., L_i}. \\
    \bu{Индукционна стъпка} Ще покажем, че най-късите пътища от \(v_0\) до всеки връх от \(L_{i + 1}\) 
    в покриващото дърво са с дължина \(i + 1\). \\
    Да допуснем, че \mexpr{\exists v \in L_{i + 1}}, такъв че \mexpr{v_0, ..., w, v} е най-къс път от 
    \(v_0\) до \(v\) с дължина \(k < i + 1\). Тогава \(v_0, ..., w\) е най-къс път от \(v_0\) до \(w\) 
    и дължината му е \(k - 1 < i\). Съгласно И.П. \(w \in L_{k - 1}\) и алгоритъмът трябваше да 
    постави \(v\) в \(L_k, k < i + 1\), което е противоречие с \(v \in L_{i + 1}\).
    Следвателно, твърдението е в сила и за \(L_{i + 1}\).
\end{proof}

\subsection*{Алгоритъм на Дейкстра.}

\begin{alg}[на Дейкстра]
    \(\newline\)Вход: свързан граф \(G = (V, E)\) с тегловна ф-я по ребрата \(w: E \to \mathbb{R}^+\) и начален 
    връх \(v_0 \in V\) \\
    Изход: дърво но минималните пътища от \(v_0\) до всички останали върхове в \(G\)
    \begin{enumerate}
        \item Разширяваме \(w: E \to \mathbb{R}^+\) до \(w^*: V \times V \to \mathbb{R}^+\)
        \item Нека \mexpr{dist[0] = 0, part[0] = -1} и \(U = \{0\}\), а \mexpr{dist[i] = w^*(0, i)} и 
        \mexpr{part[i] = 0} за \(i \in I_n\)
        \item Повтаряме \(n - 1\) пъти следните стъпки:
        \begin{enumerate}
            \item Избираме връх \(j \not \in U\), за който \(dist[j]\) е минимално и \mexpr{U = U \cup \{j\}}
            \item За всяко \(k \not \in U\) пресмятаме \mexpr{dist[k] = min(dist[k], dist[j] + w*(j, k))}. Ако
            \(min\) е \mexpr{dist[j] + w*(j, k)}, тогава \mexpr{part[k] = j}.
        \end{enumerate}
    \end{enumerate}
\end{alg}
