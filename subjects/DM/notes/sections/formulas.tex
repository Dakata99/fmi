\section{All you need to know...}

\begin{itemize}
	\item За всяко \(k \in \mathbb{N}\), броят на числата, кратни на \(m \in \mathbb{N}\) и ненадхвърлящи \(k\),
	      е \(\left\lfloor \frac{k}{m} \right\rfloor\).
\end{itemize}

\subsection*{Комбинаторика}

Нека \(n, m \in \mathbb{N}\). Тогава:
\begin{itemize}
	\item \(\binom{n}{m} = 0\), ако \(m > n\)
	\item \(\binom{n}{m} = \binom{n}{n - m}\)
	\item \(\binom{n}{m} = \binom{n - 1}{m} + \binom{n - 1}{m - 1}\)
\end{itemize}

\[
	\sum_{k = 0}^n \binom{n}{k}^2 = \binom{2n}{n}
\]

\[
	\sum_{k = 0}^{n} \binom{n}{k} = 2^n\
\]

\begin{align*}
	\sum_{k = 0}^{m} \binom{n + k - 1}{k} = \binom{n + m}{m} \\
	\text{е броят разходки в мрежа, където \(m\) са редовете, а \(n\) са колоните}
\end{align*}

Нека \(A\) и \(B\) са непразни множества, \(|A| = m, |B| = n\).

Броят на инекциите е \(f: A \rightarrow B\) е \(min\{n, m\}\).
Инекциите map-ват всеки елемент от А в уникален елемент от B.

Сумата
\[
	\sum_{k = 0}^{n} (-1)^k \binom{n}{k} (n - k)^m =
	\begin{cases}
		0,  & m < n (\implies \text{няма сюрекции}) \\
		n!, & m = n
	\end{cases}
\]
е броят на сюрекциите \(f: A \rightarrow B\) с \(m\)-елементен домейн и \(n\)-елементен кодомейн.
Ако \(n = m\), то всяка сюрекция е и инекция (\(\rightarrow\) и биекция), а те са \(n!\) на брой.

\begin{note}
	Броят на "двукратните" сюрекции (т.е. всеки елемент от кодомейна е покрит поне 2 пъти) е:
	\begin{align*}
		\sum_{k = 0}^{n} [(-1)^k \binom{n}{k} \sum_{l = 0}^{k} \{\binom{k}{l} (\prod_{t = 0}^{l - 1} (m - t)) (n - k)^{m -l}\}]
	\end{align*}
\end{note}

\begin{example}
	Нека имаме 19 различими подаръка и 6 деца (очевидно са различими). По колко начина всяко дете може
	да получи поне 2 подаръка?
\end{example}

Броя функции \(f: A \rightarrow B\), които са:
\begin{itemize}
	\item инекции е \(n.(n - 1). ... .(n - m + 1)\)
	\item не са сюрекции е "всички ф-ии без ограничение - сюрективните = "
	      \(n^m - \sum_{k = 0}^{n} (-1)^k \binom{n}{k} (n - k)^m\)
	\item инекции и не са сюрекции:
	      \begin{itemize}
		      \item ако \(m = n\), то 0, защото тогава всички инекции са и сюрекции
		      \item ако \(m < n\), то \(n.(n - 1). ... .(n - m + 1)\), защото тогава няма сюрекции
		      \item ако \(m > n\), то 0, защото тогава няма инекции
	      \end{itemize}
\end{itemize}

Нека е дадено уравнението
\[
	x_1 + ... + x_n = k
\], където \(x_1, ..., x_n \in \mathbb{N} \text{(са различими кутии), а } k \in \mathbb{N}
\text{(са неразличимите топки)}\).
Тогава броят решения без ограничения е:
\[
	\binom{n + k - 1}{n - 1} = \binom{n + k - 1}{k}
\]

\includepdf[pages=-]{./sections/12fold.pdf}

\subsection*{Графи}
