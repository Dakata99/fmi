\section{Тема 8}

% Принципи на изброителната комбинаторика: принцип на Дирихле, принцип
% на биекцията, принципи на събирането (разбиването) и изваждането,
% принцип на умножението (Декартовото произведение) и делението. 
\subsection{Принципи на изброителната комбинаторика}

\begin{principle}[на Дирихле]
	Ако \(X\) и \(Y\) са крайни множества и \mexpr{|X| > |Y|}, то не съществува инекция \mexpr{f: X \to Y}.
\end{principle}

\underline{Алтернативна формулировка}: ако има \(m\) ябълки в \(n\) чекмеджета и \(m > n\), то в поне едно
чекмедже има повече от една ябълка.

\underline{Обобщение}: ако има \mexpr{k * n + 1} ябълки в \(n\) чекмеджета, то в поне едно чекмедже има повече
от \(k\) ябълки.

\begin{principle}[на биекцията]
	Нека \(X\) и \(Y\) са крайни множества. Тогава \mexpr{|X| = |Y|} \totw съществува биекция \mexpr{f: X \to Y}.
\end{principle}

\begin{principle}[на събирането (разбиването)]
	Нека е дадено множество \(X\) и разбиване \mexpr{Y = \{Y_1, ..., Y_k\}} на \(X\). Тогава
	\mexpr{|X| = |Y_1| + ... + |Y_k|}.
\end{principle}

Това остава в сила дори някои от множествата \mexpr{Y_1, ... Y_k} да са празни, защото мощностите на
празните \(Y_i\) са нули и те не се отразяват на сумата.

\begin{principle}[на изваждането]
	Нека е дадено множество \(X\) в универсум \(U\). Тогава \mexpr{|X| = |U| - |\overline{X}|}.
\end{principle}

Очевидно \mexpr{\{X, \overline{X}\}} е разбиване на универсума, така че от принципа на събирането
имаме, че \mexpr{|U| = |X| + |\overline{X}|}.
Не е възможно \(\overline{X}\) да е празно, но дори тогава твърдението остава в сила.

\begin{principle}[на умножението]
	Нека \(X\) и \(Y\) са множества. Тогава \mexpr{|X \times Y| = |X| * |Y|.}
\end{principle}

\begin{principle}[на делението]
	Нека \(X\) е множество и \(R \subseteq X^2\) е релация на еквивалентност. Нека \(X\) има \(k\) класа
	на еквивалентност и всеки клас на еквивалентност има мощност \(m\). Тогава
	\mexpr{m = \frac{|X|}{k}}.
\end{principle}

\subsection{Принцип на включването и изключването}

\begin{principle}[на включването и изключването]
	Нека е дадено покриване на множество и търсим мощността на множеството, като събираме и изваждаме
	мощностите на дяловете на покриването, техните сечения по двойки, по тройки и т.н.

	\textbf{Обща формулировка}: За всяко \(n \ge 1\), за всеки \(n\) множества \mexpr{A_1, A_2, ..., A_n}:
	\begin{align}
		|A_1 \cup ... \cup A_{n}| = \sum_{1 \le i \le n} |A_i| & - \sum_{1 \le i < j \le n} |A_i \cap A_j|      \\ \nonumber
		                                                       & + ... + (-1)^{n - 1} |A_1 \cap ... \cap A_{n}|
	\end{align}
\end{principle}

\begin{proof}
	С индукция по \(n\):

	\bu{База}: \mexpr{n = 1}

	Тогава (1) става \mexpr{|A_1| = |A_1|}.

	\bu{Индукционно предположение}: нека твърдението е изпълнено за всеки \(n - 1\) множества, т.е.
	\begin{align}
		|A_1 \cup ... \cup A_{n - 1}| & = \sum_{1 \le i \le n - 1} |A_i| \nonumber
		- \sum_{1 \le i < j \le n - 1} |A_i \cap A_j|                                       \\
		                              & + \dots + (-1)^{n - 2}|A_1 \cap ... \cap A_{n - 1}|
	\end{align}

	\bu{Индукционна стъпка}: ще докажем твърдението за всеки \(n\) множества.

	В сила е
	\begin{align}
		 & |A_1 \cup ... \cup A_{n - 1} \cup A_n| = \nonumber
		|\underbrace{(A_1 \cup ... \cup A_{n - 1})}_\text{X} \cup \underbrace{A_n}_\text{Y}| = \nonumber \\
		 & = |\underbrace{(A_1 \cup ... \cup A_{n - 1})}_\text{X}| + |\underbrace{A_n}_\text{Y}|
		- |\underbrace{(A_1 \cup ... \cup A_{n - 1})}_\text{X} \cap \underbrace{A_n}_\text{Y}|
	\end{align}, тъй като \mexpr{|X \cup Y| = |X| + |Y| - |X \cap Y|} (от (1) при \mexpr{n = 2}).

	Знаем колко е \mexpr{|A_1 \cup \dots \cup A_{n - 1}|} от (2).

	Да разгледаме \mexpr{|(A_1 \cup \dots \cup A_{n - 1}) \cap A_n|}. В сила е
	\begin{align}
		(A_1 \cup \dots \cup A_{n - 1}) \cap A_n = (A_1 \cap A_n) \cup \dots \cup (A_{n - 1} \cap A_n)
	\end{align}
	заради дистрибутивността на сечението спрямо обединението. Дясната страна на (4) е обобщение на
	\(n - 1\) множества и (2) е приложимо, т.е.:
	\begin{align}
		 & |(A_1 \cap A_n) \cup \dots \cup (A_{n - 1} \cap A_n)| =                                     \\ \nonumber
		 & \sum_{1 \le i \le n - 1} |A_i \cap A_n| \nonumber
		- \sum_{1 \le i < j \le n - 1} |(A_i \cap A_n) \cap (A_j \cap A_n)|                            \\ \nonumber
		 & + \sum_{1 \le i < j < k \le n - 1} |(A_i \cap A_n) \cap (A_j \cap A_n) \cap (A_k \cap A_n)| \\ \nonumber
		 & + \dots + (-1)^{n - 2} |(A_1 \cap A_n) \cap \dots \cap (A_{n - 1} \cap A_n)|
	\end{align}
	Получаваме
	\begin{align}
		|(A_1 \cup \dots \cup A_{n - 1}) \cap A_n| & = \sum_{1 \le i \le n - 1} |A_i \cap A_n|                           \\ \nonumber
		                                           & - \sum_{1 \le i < j \le n - 1} |A_i \cap A_j \cap A_n|              \\ \nonumber
		                                           & + \sum_{1 \le i < j < k \le n - 1} |A_i \cap A_j \cap A_k \cap A_n| \\ \nonumber
		                                           & \dots                                                               \\ \nonumber
		                                           & + (-1)^{n - 2} |A_1 \cap A_2 \cap \dots \cap A_{n - 1} \cap A_n|
	\end{align}
	От (2) и (6) получаваме
	\begin{align*}
		|A_1 \cup \dots \cup A_{n - 1} \cup A_n| =                                                                 \\
		                           & (\sum_{1 \le i \le n - 1} |A_i| - \sum_{1 \le i < j \le n - 1} |A_i \cap A_j| \\
		                           & + \dots + (-1)^{n - 2} |A_1 \cap \dots \cap A_{n - 1}|)                       \\
		+ |A_n|                                                                                                    \\
		                           & - (\sum_{1 \le i \le n - 1} |A_i \cap A_n|                                    \\
		                           & - \dots + (-1)^{n - 2} |A_1 \cap A_2 \cap \dots \cap A_{n - 1} \cap A_n|) =   \\
		\sum_{1 \le i \le n} |A_i| & - \sum_{1 \le i < j \le n - 1} |A_i \cap A_j|                                 \\
		                           & + \dots + (-1)^{n - 2} |A_1 \cap \dots \cap A_{n - 1}|                        \\
		                           & - \sum_{1 \le i \le n - 1} |A_i \cap A_n|                                     \\
		                           & + \dots + (-1)^{n - 1} |A_1 \cap A_2 \cap \dots \cap A_{n - 1} \cap A_n|
	\end{align*}
	Групираме събираемите по подходящ начин:
	\begin{align*}
		|A_1 \cup \dots \cup A_{n - 1} \cup A_n| =                                                                                             \\
		\sum_{1 \le i \le n} |A_i| & - \underbrace{\sum_{1 \le i < j \le n - 1} |A_i \cap A_j|}_\text{*}
		+ \underbrace{\sum_{1 \le i < j < k \le n - 1} |A_i \cap A_j \cap A_k|}_\text{**}                                                      \\
		                           & + \dots + \underbrace{(-1)^{n - 2} |A_1 \cap \dots \cap A_{n - 1}|}_\text{***}                            \\
		                           & - \underbrace{\sum_{1 \le i \le n - 1} |A_i \cap A_n|}_\text{*}
		+ \underbrace{\sum_{1 \le i < j \le n - 1} |A_i \cap A_j \cap A_n|}_\text{**}                                                          \\
		                           & - \dots +                                                                                                 \\
		                           & + \underbrace{(-1)^{n - 2} \sum_{1 \le i_1 < \dots < i_{n - 2} \le n - 1}
		|A_{i_1} \cap \dots \cap A_{i_{n - 2}} \cap A_n|}_\text{***}                                                                           \\
		                           & + (-1)^{n - 1} |A_1 \cap \dots \cap A_n| =                                                                \\ \sum_{1 \le i \le n} |A_i|
		                           & - \sum_{1 \le i < j \le n} |A_i \cap A_j| + \sum_{1 \le i < j \le n} |A_i \cap A_j \cap A_k|              \\
		                           & + \dots + (-1)^{n - 2} \sum_{1 \le i_1 < \dots < i_{n - 1} \le n} |A_{i_1} \cap \dots \cap A_{i_{n - 1}}| \\
		                           & + (-1)^{n - 1} |A_1 \cap \dots \cap A_n| = (1)
	\end{align*}
\end{proof}

\begin{consequence}
	За всяко \(n \ge 1\), за всеки \(n\) множества \(A_1, A_2, \dots, A_n\), намиращи се в произволен
	универсум \(U\):
	\begin{align}
		|\overline{A_1}\cap \dots \cap \overline{A_n}| = |U| & - \sum_{1 \le i \le n} |A_i| \nonumber
		+ \sum_{1 \le i < j \le n} |A_i \cap A_j|                                                         \\
		                                                     & - \dots + (-1)^n |A_1 \cap \dots \cap A_n|
	\end{align}
\end{consequence}

\begin{proof}
	Имаме \mexpr{|\overline{A_1 \cup \dots \cup A_n|} = |U| - |A_1 \cup \dots \cup A_n|} от принципа на
	изваждането.
	Тогава \mexpr{|\overline{A_1 \cup \dots \cup A_n|} = |\overline{A_1}\cap \dots \cap \overline{A_n}|
		= |U| - (1)} и получаваме точно (7).
\end{proof}
